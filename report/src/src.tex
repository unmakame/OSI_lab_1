\section{Метод решения}
Данная программа реализует многопроцессную обработку текстовых данных с использованием каналов (pipes) для межпроцессного взаимодействия.
Основной алгоритм: родительский процесс читает строки из стандартного ввода и направляет строки длиной более 10 символов во второй дочерний процесс, остальные - в первый. Каждый дочерний процесс получает строки из своего канала, удаляет гласные буквы и выводит результат в стандартный вывод.

Ключевые компоненты:\\
ParentProcess - управляет каналами и дочерними процессами\\
PipeManager - кросс-платформенная реализация каналов\\
StringProcessor - обработка строк (удаление гласных и фильтрация по длине)\\
Platform - кроссплатформенные системные функции\\

Системные вызовы:\\
Windows: CreatePipe, CreateProcess, ReadFile, WriteFile\\
Linux: pipe, fork, execl, read, write\\

Программа использует объектно-ориентированный подход с инкапсуляцией платформо-зависимых особенностей, что обеспечивает кроссплатформенность и четкое разделение ответственности между модулями.

\section{Описание программы}
Программа реализует многопроцессную обработку текстовых данных через каналы (pipes). \\
Родительский процесс читает строки из стандартного ввода и распределяет их между двумя дочерними процессами согласно варианту 17: строки длиной более 10 символов отправляются во второй процесс (pipe2), остальные - в первый процесс (pipe1). \\
Каждый дочерний процесс удаляет все гласные буквы из полученных строк и выводит результат в стандартный вывод.\\

Архитектура программы включает несколько модулей. \\
В main.cpp находится точка входа, создающая ParentProcess и запрашивающая имена файлов для дочерних процессов. \\
Класс ParentProcess (parentProcess.cpp) управляет всей работой: создает каналы, запускает дочерние процессы и распределяет данные по длине строк. \\
Класс PipeManager (pipeManager.cpp) инкапсулирует работу с каналами, используя CreatePipe на Windows и pipe на Linux. \\
Класс StringProcessor (stringProcessor.cpp) реализует логику обработки строк: удаление гласных и проверку длины для фильтрации. \\
Platform (platform.cpp) предоставляет кроссплатформенные функции для работы с процессами и временными задержками.\\

Дочерние процессы (childProcess.cpp) являются отдельными исполняемыми файлами, которые получают строки через стандартный ввод, обрабатывают их и выводят результаты в стандартный вывод, соответствуя требованию условия.
\section{Исходный код}

\subsection{Основные модули}

\textbf{main.cpp} - точка входа программы:
\lstinputlisting[language=C++,caption={main.cpp}]{../../src/main.cpp}

\textbf{parentProcess.cpp} - родительский процесс:
\lstinputlisting[language=C++,caption={parentProcess.cpp}]{../../src/parentProcess.cpp}

\textbf{parentProcessImplementation.cpp} - реализация родительского процесса:
\lstinputlisting[language=C++,caption={parentProcessImplementation.cpp}]{../../src/parentProcessImplementation.cpp}

\textbf{childProcess.cpp} - дочерний процесс:
\lstinputlisting[language=C++,caption={childProcess.cpp}]{../../src/childProcess.cpp}

\subsection{Вспомогательные модули}

\textbf{pipeManager.cpp} - управление каналами:
\lstinputlisting[language=C++,caption={pipeManager.cpp}]{../../src/pipeManager.cpp}

\textbf{stringProcessor.cpp} - обработка строк:
\lstinputlisting[language=C++,caption={stringProcessor.cpp}]{../../src/stringProcessor.cpp}

\textbf{platform.cpp} - кроссплатформенные функции:
\lstinputlisting[language=C++,caption={platform.cpp}]{../../src/platform.cpp}


\section{Логи выполнения программы}

\begin{lstlisting}[caption={Логи выполнения программы}]
=== Parent Process Started ===
ParentProcess constructor: pipes created
=== Starting Parent Process ===
Enter filename for child1: file1.txt
Enter filename for child2: file2.txt
Creating child process 1...
Pipe created successfully (read: 3, write: 4)
Starting child process 1 with file: file1.txt
Linux: Forking child process...
Linux: Child process created with PID: 274150
Child process 1 started successfully
Closed read end of pipe1 in parent
Creating child process 2...
Pipe created successfully (read: 5, write: 6)
Starting child process 2 with file: file2.txt
Linux: Forking child process...
Linux: Child process created with PID: 274151
Child process 2 started successfully
Closed read end of pipe2 in parent
Enter lines (empty line to end):
hello
Sending line to child 1 (length: 5): hello
Pipe wrote 5 bytes
Pipe wrote 1 bytes
very long string
Sending line to child 2 (length: 16): very long string
Pipe wrote 16 bytes
Pipe wrote 1 bytes
test
Sending line to child 1 (length: 4): test
Pipe wrote 4 bytes
Pipe wrote 1 bytes

Total lines sent: 3
Closing write ends of pipes...
Linux: Closing pipe handle 4
Linux: Closing pipe handle 6
Waiting for child processes to finish...
Linux: Waiting for process 274150 to finish...
=== Child Process Started ===
Output filename: file1.txt
File opened successfully
Linux: Child process executing: ./childProcess
Linux: Duplicating fd 3 to 0
Linux: Closing pipe handle 3
Received line: 'hello'
Reversing string: 'hello'
Reversed string: 'olleh'
Written to file: 'olleh'
olleh
Received line: 'test'
Reversing string: 'test'
Reversed string: 'tset'
Written to file: 'tset'
tset
Child process finished. Processed 2 lines.
=== Child Process Finished ===
Linux: Process 274150 finished with status: 0
Child process 1 finished
Linux: Waiting for process 274151 to finish...
=== Child Process Started ===
Output filename: file2.txt
File opened successfully
Linux: Child process executing: ./childProcess
Linux: Duplicating fd 5 to 0
Linux: Closing pipe handle 5
Received line: 'very long string'
Reversing string: 'very long string'
Reversed string: 'gnirts gnol yrev'
Written to file: 'gnirts gnol yrev'
gnirts gnol yrev
Child process finished. Processed 1 lines.
=== Child Process Finished ===
Linux: Process 274151 finished with status: 0
Child process 2 finished
Parent: all children finished successfully.
=== Parent Process Finished ===
ParentProcess destructor: cleaning up resources
\end{lstlisting}

\section{Системные вызовы}

\lstinputlisting[caption={Системные вызовы (strace)}]{strace.log}



Программа демонстрирует корректную работу механизма межпроцессного взаимодействия через именованные каналы и правильное распределение строк между процессами согласно варианту 17.