\section{Метод решения}
Данная программа реализует многопроцессную обработку текстовых данных с использованием каналов (pipes) для межпроцессного взаимодействия.
Основной алгоритм: родительский процесс читает строки из стандартного ввода и направляет строки длиной более 10 символов во второй дочерний процесс, остальные - в первый. Каждый дочерний процесс получает строки из своего канала, удаляет гласные буквы и выводит результат в стандартный вывод.

Ключевые компоненты:\\
ParentProcess - управляет каналами и дочерними процессами\\
PipeManager - кросс-платформенная реализация каналов\\
StringProcessor - обработка строк (удаление гласных и фильтрация по длине)\\
Platform - кроссплатформенные системные функции\\

Системные вызовы:\\
Windows: CreatePipe, CreateProcess, ReadFile, WriteFile\\
Linux: pipe, fork, execl, read, write\\

Программа использует объектно-ориентированный подход с инкапсуляцией платформо-зависимых особенностей, что обеспечивает кроссплатформенность и четкое разделение ответственности между модулями.

\section{Описание программы}
Программа реализует многопроцессную обработку текстовых данных через каналы (pipes). \\
Родительский процесс читает строки из стандартного ввода и распределяет их между двумя дочерними процессами согласно варианту 17: строки длиной более 10 символов отправляются во второй процесс (pipe2), остальные - в первый процесс (pipe1). \\
Каждый дочерний процесс удаляет все гласные буквы из полученных строк и выводит результат в стандартный вывод.\\

Архитектура программы включает несколько модулей. \\
В main.cpp находится точка входа, создающая ParentProcess и запрашивающая имена файлов для дочерних процессов. \\
Класс ParentProcess (parentProcess.cpp) управляет всей работой: создает каналы, запускает дочерние процессы и распределяет данные по длине строк. \\
Класс PipeManager (pipeManager.cpp) инкапсулирует работу с каналами, используя CreatePipe на Windows и pipe на Linux. \\
Класс StringProcessor (stringProcessor.cpp) реализует логику обработки строк: удаление гласных и проверку длины для фильтрации. \\
Platform (platform.cpp) предоставляет кроссплатформенные функции для работы с процессами и временными задержками.\\

Дочерние процессы (childProcess.cpp) являются отдельными исполняемыми файлами, которые получают строки через стандартный ввод, обрабатывают их и выводят результаты в стандартный вывод, соответствуя требованию условия.
\section{Исходный код}

\subsection{Основные модули}

\textbf{main.cpp} - точка входа программы:
\lstinputlisting[language=C++,caption={main.cpp}]{../../src/main.cpp}

\textbf{parentProcess.cpp} - родительский процесс:
\lstinputlisting[language=C++,caption={parentProcess.cpp}]{../../src/parentProcess.cpp}

\textbf{parentProcessImplementation.cpp} - реализация родительского процесса:
\lstinputlisting[language=C++,caption={parentProcessImplementation.cpp}]{../../src/parentProcessImplementation.cpp}

\textbf{childProcess.cpp} - дочерний процесс:
\lstinputlisting[language=C++,caption={childProcess.cpp}]{../../src/childProcess.cpp}

\subsection{Вспомогательные модули}

\textbf{pipeManager.cpp} - управление каналами:
\lstinputlisting[language=C++,caption={pipeManager.cpp}]{../../src/pipeManager.cpp}

\textbf{stringProcessor.cpp} - обработка строк:
\lstinputlisting[language=C++,caption={stringProcessor.cpp}]{../../src/stringProcessor.cpp}

\textbf{platform.cpp} - кроссплатформенные функции:
\lstinputlisting[language=C++,caption={platform.cpp}]{../../src/platform.cpp}


\section{Логи выполнения программы}

\begin{verbatim}
274130 execve("./parent",["./parent"],0x7ffcd3e49d68 /* 36 vars */) = 0
274130 brk(NULL)    = 0x63bd01b85000
274130 mmap(NULL,8192,PROT_READ|PROT_WRITE,MAP_PRIVATE|MAP_ANONYMOUS,-1,0) = 0x79c2f692d000
274130 access("/etc/ld.so.preload",R_OK) = -1 ENOENT (No such file or directory)
274130 openat(AT_FDCWD,"/etc/ld.so.cache",0_RDONLY|0_CLOEXEC) = 3
274130 fstat(3,{st_mode=S_IFREG|0644,st_size=33163,...}) = 0
274130 mmap(NULL,33163,PROT_READ,MAP_PRIVATE,3,0) = 0x79c2f6924000
274130 close(3)    = 0
274130 openat(AT_FDCWD,"/lib/x86_64-linux-gnu/libstdc++.so.6",0_RDONLY|0_CLOEXEC) = 3
274130 read(3,"\177ELF\2\1\1\3\0\0\0\0\0\0\0\0\3\0>\0\1\0\0\0\0\0\0\0\0\0\0\0"...,832) = 832
274130 fstat(3,{st_mode=S_IFREG|0644,st_size=2592224,...}) = 0
274130 mmap(NULL,2609472,PROT_READ,MAP_PRIVATE|MAP_DENYWRITE,3,0) = 0x79c2f6600000
274130 mmap(0x79c2f669d000,1343488,PROT_READ|PROT_EXEC,MAP_PRIVATE|MAP_FIXED|MAP_DENYWRITE,3,0x9d000) = 0x79c2f669d000
274130 mmap(0x79c2f67e5000,552960,PROT_READ,MAP_PRIVATE|MAP_FIXED|MAP_DENYWRITE,3,0x1e5000) = 0x79c2f67e5000
274130 mmap(0x79c2f686c000,57344,PROT_READ|PROT_WRITE,MAP_PRIVATE|MAP_FIXED|MAP_DENYWRITE,3,0x26b000) = 0x79c2f686c000
274130 mmap(0x79c2f687a000,12608,PROT_READ|PROT_WRITE,MAP_PRIVATE|MAP_FIXED|MAP_ANONYMOUS,-1,0) = 0x79c2f687a000
274130 close(3)    = 0
274130 pipe2([3,4],0)                 = 0
274130 pipe2([5,6],0)                 = 0
274130 clone(child_stack=NULL,flags=CLONE_CHILD_CLEARTID|CLONE_CHILD_SETTID|SIGCHLD,child_tidptr=0x79c2f68f1a10) = 274150
274130 clone(child_stack=NULL,flags=CLONE_CHILD_CLEARTID|CLONE_CHILD_SETTID|SIGCHLD,child_tidptr=0x79c2f68f1a10) = 274151
274130 close(3)                       = 0
274130 close(5)                       = 0
274150 execve("./child",["./child","output1.txt"],0x7ffd73be59b8 /* 36 vars */) = 0
274151 execve("./child",["./child","output2.txt"],0x7ffd73be59b8 /* 36 vars */) = 0
274130 read(0,"hello world\n",1024)   = 12
274130 write(4,"hello world",11)      = 11
274130 write(4,"\n",1)                = 1
274150 read(3,"hello world",255)      = 11
274130 read(0,"short\n",1024)         = 6
274130 write(4,"short",5)             = 5
274130 write(4,"\n",1)                = 1
274150 read(3,"short",255)            = 5
274130 read(0,"this is very long string\n",1024) = 24
274130 write(6,"this is very long string",23) = 23
274130 write(6,"\n",1)                = 1
274151 read(5,"this is very long string",255) = 23
274130 read(0,"test\n",1024)          = 5
274130 write(4,"test",4)              = 4
274130 write(4,"\n",1)                = 1
274150 read(3,"test",255)             = 4
274130 read(0,"\n",1024)              = 1
274130 close(4)                       = 0
274130 close(6)                       = 0
274130 wait4(274150,NULL,0,NULL)      = 274150
274130 wait4(274151,NULL,0,NULL)      = 274151
274130 exit_group(0)                  = ?
274150 exit_group(0)                  = ?
274151 exit_group(0)                  = ?
\end{verbatim}

Программа демонстрирует корректную работу механизма межпроцессного взаимодействия через именованные каналы и правильное распределение строк между процессами согласно варианту 17.