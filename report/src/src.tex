\section{Метод решения}
Данная программа реализует многопроцессную обработку текстовых данных с использованием каналов (pipes) для межпроцессного взаимодействия.
Основной алгоритм: родительский процесс читает строки из стандартного ввода и направляет строки длиной более 10 символов во второй дочерний процесс, остальные - в первый. Каждый дочерний процесс получает строки из своего канала, удаляет гласные буквы и выводит результат в стандартный вывод.

Ключевые компоненты:\\
ParentProcess - управляет каналами и дочерними процессами\\
PipeManager - кросс-платформенная реализация каналов\\
StringProcessor - обработка строк (удаление гласных и фильтрация по длине)\\
Platform - кроссплатформенные системные функции\\

Системные вызовы:\\
Windows: CreatePipe, CreateProcess, ReadFile, WriteFile\\
Linux: pipe, fork, execl, read, write\\

Программа использует объектно-ориентированный подход с инкапсуляцией платформо-зависимых особенностей, что обеспечивает кроссплатформенность и четкое разделение ответственности между модулями.

\section{Описание программы}
Программа реализует многопроцессную обработку текстовых данных через каналы (pipes). \\
Родительский процесс читает строки из стандартного ввода и распределяет их между двумя дочерними процессами согласно варианту 17: строки длиной более 10 символов отправляются во второй процесс (pipe2), остальные - в первый процесс (pipe1). \\
Каждый дочерний процесс удаляет все гласные буквы из полученных строк и выводит результат в стандартный вывод.\\

Архитектура программы включает несколько модулей. \\
В main.cpp находится точка входа, создающая ParentProcess и запрашивающая имена файлов для дочерних процессов. \\
Класс ParentProcess (parentProcess.cpp) управляет всей работой: создает каналы, запускает дочерние процессы и распределяет данные по длине строк. \\
Класс PipeManager (pipeManager.cpp) инкапсулирует работу с каналами, используя CreatePipe на Windows и pipe на Linux. \\
Класс StringProcessor (stringProcessor.cpp) реализует логику обработки строк: удаление гласных и проверку длины для фильтрации. \\
Platform (platform.cpp) предоставляет кроссплатформенные функции для работы с процессами и временными задержками.\\

Дочерние процессы (childProcess.cpp) являются отдельными исполняемыми файлами, которые получают строки через стандартный ввод, обрабатывают их и выводят результаты в стандартный вывод, соответствуя требованию условия.